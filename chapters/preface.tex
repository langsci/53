\addchap{Preface}

This book contributes to our understanding of the origins of spatial language
by carrying out language game experiments with artificial agents instantiated as
humanoid robots. It tests the theory of language evolution by linguistic selection,
which states that language emerges through a cultural process based on the recruitment of
various cognitive capacities in the service of language. Agents generate possible
paradigmatic choices in their language systems and explore different language strategies.
Which ones survive and dominate depends on linguistic selection criteria, such
as expressive adequacy with respect to the ecological challenges and conditions in the
environment, minimization of cognitive effort, and communicative success.

To anchor this case study in empirical phenomena,
the book reconstructs the syntax and semantics of German spatial language,
in particular German locative phrases. 
Syntactic processing is organized using Fluid Construction Grammar (FCG), 
a computational formalism for representing
linguistic knowledge. For the semantics the book focusses in particular
on proximal, projective and absolute spatial categories as well as perspective,
perspective reversal and frame of reference. The semantic investigations
use the perspective of Embodied Cognitive Semantics.
The spatial semantics is grounded in the sensory-motor experiences of the robot
and made compositional by using the Incremental Recruitment Language (IRL) developed
for this purpose. The complete reconstructed system allows humanoid robots to communicate successfully and efficiently using the German locative system and 
provides a performance base line. The reconstruction shows that the computational
formalisms, i.e. FCG and IRL, are sufficient for tackling complex natural language phenomena.
Moreover, the reconstruction efforts reveal the tight interaction of syntax and semantics 
in German locative phrases. 

The second part of the book concentrates on the evolution of spatial language.
First the focus is on the formation and acquisition of 
spatial language by proposing strategies in the form of invention, adoption, 
and alignment operators. The book shows the adequacy of these strategies 
in acquisition experiments in which some agents act as learners
and others as tutors. It shows next in language formation experiments
that these strategies are sufficient to allow a population to self-organize 
a spatial language system from scratch.
The book continues by studying the origins and competition of 
language strategies. Different conceptual strategies are considered and studied 
systematically, particularly in relation to
the properties of the environment, for example, whether a global landmark is available.
Different linguistic strategies are studied as well,
for instance, the problem of choosing a particular reference object on the scene can be solved
by the invention of markers, which allows many different reference objects, or by converging
to a standard single reference object, such as a global landmark.

The book demonstrates that the theory of language evolution 
by linguistic selection leads to operational experiments in which artificial agents 
self-organize semantically rich and syntactically complex language. Moreover, 
many issues in cognitive science, ranging from perception and conceptualization to
language processing, had to be dealt with to instantiate this theory, so that this book 
contributes not only to the study of language evolution but to the investigation 
of the cognitive bases of spatial language as well.

This book would not have been possible without the hard work of
the people at Sony Computer Science Laboratory Paris and the A.I. Lab at 
the Vrije Universiteit Brussels. Many of them have left traces in software and 
ideas that provide the background against which a book like this one becomes 
possible. Most notably I would like to thank the current and past members
of the AI lab in Brussels and Sony CSL Paris who I have met
and who have made contributions to the various software systems that underly the 
experiments described in this book: Katrien Beuls, Joris Bleys,
Joachim De Beule, Wouter van den Broeck, Remi van Trijp and Pieter Wellens. 

Martin Loetzsch and Simon Pauw had big impact on many issues discussed in this book. 
I am indebted to all of them for long discussions that have tremendously shaped my way of thinking 
and for their collaboration on different aspects of spatial language, conceptualization and embodiment.

Last but not least, I would like to thank Luc Steels who has had tremendous impact
on the intellectual ideas put forth in this book, provided the necessary environment 
to conduct this research, and who continues to be an inspirational and visionary figure
for future work.