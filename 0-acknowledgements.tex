Certainly, this thesis would not have been possible without a
wide range of stable usable software systems such as
the Babel2 framework and Fluid Construction Grammar (FCG)
used in the experiments. I thank all people involved in developing these
systems. Most notably I would like to thank the current and past members
of the AI lab in Brussels and Sony CSL Paris who I have met
and who have made contributions to these systems: Joris Bleys,
Joachim De Beule, Martin Loetzsch, Remi van Trijp and Pieter Wellens. 
It has been outstanding to work with all of you.

Working with real robots is a tedious but fun task which I have 
much enjoyed because of the collaboration with Martin Loetzsch 
and Remi van Trijp in Tokyo. Martin Loetzsch also contributed
substantially to the vision system used in the experiments.
The robots used in the experiments were graciously provided
by Masahiro Fujita, Hideki Shimomura and their team at
Sony Corporation in Tokyo. I am grateful for their support.

As part of this thesis, the Incremental Recruitment Language (IRL) 
was developed and enhanced. There are number of people who have 
worked on IRL. Wouter van den Broeck developed an early version of the system.
Joris Bleys, Martin Loetzsch and Simon Pauw provided many
ideas and implementations making the system as usable as it is today.
I thank all of them for their help and collaboration.

Martin Loetzsch and Simon Pauw had big impact
on many issues discussed in this thesis. I am indebted to both
of them for long discussions that have tremendously shaped my way of thinking 
and for their collaboration on different aspects of spatial language, 
conceptualization and embodiment. Above all, I thank both of them 
for the fun and excitement that characterize the last three 
years of my life.

There are a number concrete contributions that
helped to make this thesis. I thank Kateryna Gerasymova
and Nancy Chang for their helpful comments on earlier versions of
the thesis. I thank Joris Bleys for providing the TEX-macro
for semantic operations and Remi van Trijp for his translation
of the summary into the funny Flemish language. Moreover, I much
appreciate Lara Mennes help in dealing with the university
administration and Katrien Beuls great efforts in getting the thesis
printed when I was not in Brussels. 

I am indebted to the members of my PhD committee, in particular, 
Tony Belpaeme, Thora Tenbrink as external reviewers and Tom Lenaerts, 
Bernard Manderick, Anne Nowe and Beat Signer as internal reviewers
who have provided valuable comments and helped improve this work
substantially. I am extremely grateful to Luc Steels for supervising this thesis 
and mentoring my intellectual development. Most if not all fundamental 
ideas put forward in this thesis are based on his extraordinarily rich 
and exciting work.

I would also like to thank my family, especially my
parents, grandparents, my brother and his family for
their immense and unconditional support. 
Finally, I could not have become somebody
who finishes a thesis without Katya.

The research reported in this thesis has been financially supported 
by Sony CSL Paris and Tokyo as well as the ECAgents (FP6) and 
ALEAR (FP7) projects.